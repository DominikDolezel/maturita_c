\documentclass{memoir}
\usepackage{fullpage}
\usepackage[czech]{babel}
\usepackage{amsmath}
\usepackage{amsfonts}

\begin{document}
\section*{Stavba souvětí}
\begin{itemize}
\item poměry mezi souřadně spojenými větami
    \begin{itemize}
    \item slučovací (např. spojky \textit{a}, \textit{i}, \textit{ani})
    \item stupňovací (např. spojky \textit{dokonce, ba i})
    \item odporovací (např. spojky ale, přesto, avšak)
    \item vylučovací (např. spojky nebo, anebo)
    \item důvodový (např. spojky neboť, totiž, vždyť, přece)
    \item důsledkový (např. spojky proto, tedy, tudíž)
    \end{itemize}
    \item věta dvojčlenná = má podmět a přísudek \hfill \textit{Máme hodně úkolů.}
   	\item věta jednočlenná = má pouze přísudek, nemá podmět \hfill \textit{Prší. Je mi smutno.}
   	\item větný ekvivalent = nemá ani podmět, ani přísudek \hfill \textit{Nevstupovat!}
   	\begin{itemize}
    \item druhy vedlejších vět
    \begin{itemize}
    \item podmětná \hfill \textit{Kdo, co?}
   	\item přísudková \hfill \textit{Jaký byl/nebyl podmět?}
   	\item předmětná \hfill \textit{pádové otázky kromě 1. a 5. pádu}
   	\item přívlastková \hfill \textit{Jaký, který, čí?}
   	\item doplňková \hfill \textit{V jaké situaci?}
   	\item příslovečná
    \begin{itemize}
    \item místní \hfill \textit{Kde?}
   	\item časová \hfill \textit{Kdy?}
   	\item způsobová \hfill \textit{Jak?}
   	\item měrová \hfill \textit{Do jaké míry}
   	\item příčinná \hfill \textit{Proč?}
   	\item účelová \hfill \textit{Za jakým účelem?}
   	\item podmínková \hfill \textit{Za jakých podmínek?}
   	\item přípustková \hfill \textit{I přes co?}
    \end{itemize}
    \end{itemize}
    \end{itemize}
   	\item souvětí souřadné = spojení minimálně dvou vět hlavních
   	\item souvětí podřadné = spojení věty hlavní jednou enbo více větami vedlejšími
\end{itemize}

\section*{Větné členy}
\begin{itemize}
\item základní
\begin{itemize}
\item podmět: vyjádřený, nevyjádřený a všeobecný; holý, rozivý a několikanásobný
\item přísudek: slovesný jednoduchý (tvořen jedním slovesem v určité osobě), slovesný složený (tvořen způsobovým nebo fázovým slovesem a infinitivem); jmenný se sponou; jmenný beze spony; citoslovečný
\end{itemize}
\item rozvíjející
\begin{itemize}
\item přívlastek: shodný (shoduje se s podstatným jménem v pádě, čísle a rodě), neshodný (neshoduje se s podstatným jménem v pádě, čísle a rodě); volný (nezužuje význam řídícího podstatného jména, jen doplňuje, odděluje se čárkou), těsný (zužuje význam řídícího podsatného jména, neodděluje se čárkou), přístavek (základem je podstatné jméno nebo zájmeno, odděluje se čárkou z obou stran, pokud ale vyjadřuje funkci / titul / zaměstnání, čárkou se neodděluje); několikanásobný (několik přívlastků, jejich pořadí lze měnit, jsou odděleny čárkou nebo souřadící spojkou), postupně rozvíjející (pořadí členů není možné měnit, neoddělují se čárkou)
\item předmět
\item příslovečné určení: místa, času, způsobu, míry, příčiny, účelu, podmínky, přípustky
\item doplněk: závisí na dvou členech -- na slovesu a jméně
\end{itemize}
\end{itemize}

\section*{Slovotvorný a morfematický rozbor}
vědět co je co, morfematický rozseká slova na ty jednotlivé části, slovotvorný jde jen o krok dopředu,


\section*{Pravopis}
cvičení odhalte chyby, s/z

malá a velká písmena – prakticky – jaká jsou pravidla vysvětlit

jakoby x jako by: jakoby se dá nahradit jen jako, např. byl jakoby zmatený, když měním osobu z jako by bude jako bych

duál a duálové koncovky (očima, rukama, nohama, oběma dvěma…) – co to je


\section*{Slovní druhy}
vyjmenujte druhy zájmen – osobní, ukazovací, vztažná, tázací, neurčitá vztažná…, číslovek –  základní, řadové, druhové, násobné; určité a neurčité,..., určování slovních druhů ve větě

předložky: vlastní – můžou být jenom předložky (k, s, u, …) x nevlastní – mohou být i jiné slovní druhy (během, kolem…) x nepůvodní (vznikají z různých slovních druhů, ale tvoří vlastní slovní celek jakožto předložku, součástí často bývá podstatné jméno; např.: s ohledem na…, bez ohledu na, s přihlédnutím k…, za účelem…, pro potřeby)


\section*{Mluvnické kategorie}
co se určuje u 1.-4. – pád číslo rod vzor; co se určuje u sloves – osoba číslo způsob čas, co to je vid, …


\section*{Stupňování}
přídavných jmen a příslovcí, jak se tvoří, kolik je těch stupňů

\section*{Slovotvorba}
a to patří pod to obohacování: skládání, odvozování, zkracování, značky

\textit{obohacování spisovné slovní zásoby} češtiny (přejímání z jiných jazyků, přejímání z nespisovné češtiny, slovotvorba, přenášení významů – nové významy na již existující slova ...)

\section*{Interpunkční znaménka a jejich funkce}
interpunkční znaménka a jejich funkce: čárky tečky pomlčky spojovníky, uvozovky, středníky

\section*{větná kondenzace}
větná kondenzace: “s ohledem na zastavení činnosti”, je to v administrativním, odborném stylu, často spojeno s nepůvodními předložkami, nahromadění především podstatných jmen, místo vedlejší věty, často ve druhém pádě, často podstatná jména, která jsou původně slovesa

\section*{útvary národního j}
útvary národního jazyka: spisovné x nespisovné, co do toho patří (vysoká formální

\section*{hláskosloví}
fonetika, fonologie, ortoepie, ortofonie x typy asimilací = spodob, koncová asimilace – had [hat], spodoba (asimilace) znělosti

\section*{univerbizace a multiverbizace}

\section*{lexikologie}
jednoduché x složené pojmenování, sousloví, historismy, archaismy, slova přejatá, expresivní výrazy, deminutiva = zdrobněliny, …

\section*{práce s citacemi}
práce s citacemi: co to je, proč se používají, jak se s nimi nejlépe pracuje (citace.com), psaní zdrojů, podle norem

\section*{druhy tázacích vět}
druhy tázacích vět: jednoduché, zjišťovací (ano/ne), doplňovací (otevřená odpověď)?

\section*{typografie}
použití kurzívy, tučného písma, podtrhávání, souvisí s publicistikou, velikost písma

\section*{Automatizace a aktu}
automatizace a aktualizace výrazu: co to je

\section*{enumerace}
enumerace (výčty) - proč to autoři někdy používají

\section*{druhy vět}
druhy vět podle postoje mluvčího: oznamovací, tázací, zvolací, rozkazovací

\section*{př spřežky}
příslovečné spřežky: spojení předložky a něčeho: dohromady x do hromady, skočil do hromady sena x skočili dohromady, na shledanou není spřežka!!!

\section*{anakolut, zeugma, kontaminace, atrakce}


\end{document}
