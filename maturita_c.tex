\documentclass{memoir}
\usepackage{fullpage}
\usepackage[czech]{babel}
\usepackage{amsmath}
\usepackage{amsfonts}

\begin{document}
\section*{Stavba souvětí}
\begin{itemize}
\item poměry mezi souřadně spojenými větami
    \begin{itemize}
    \item slučovací (např. spojky \textit{a}, \textit{i}, \textit{ani})
    \item stupňovací (např. spojky \textit{dokonce, ba i})
    \item odporovací (např. spojky ale, přesto, avšak)
    \item vylučovací (např. spojky nebo, anebo)
    \item důvodový (např. spojky neboť, totiž, vždyť, přece)
    \item důsledkový (např. spojky proto, tedy, tudíž)
    \end{itemize}
    \item věta dvojčlenná = má podmět a přísudek \hfill \textit{Máme hodně úkolů.}
   	\item věta jednočlenná = má pouze přísudek, nemá podmět \hfill \textit{Prší. Je mi smutno.}
   	\item větný ekvivalent = nemá ani podmět, ani přísudek \hfill \textit{Nevstupovat!}
   	\begin{itemize}
    \item druhy vedlejších vět
    \begin{itemize}
    \item podmětná \hfill \textit{Kdo, co?}
   	\item přísudková \hfill \textit{Jaký byl/nebyl podmět?}
   	\item předmětná \hfill \textit{pádové otázky kromě 1. a 5. pádu}
   	\item přívlastková \hfill \textit{Jaký, který, čí?}
   	\item doplňková \hfill \textit{V jaké situaci?}
   	\item příslovečná
    \begin{itemize}
    \item místní \hfill \textit{Kde?}
   	\item časová \hfill \textit{Kdy?}
   	\item způsobová \hfill \textit{Jak?}
   	\item měrová \hfill \textit{Do jaké míry}
   	\item příčinná \hfill \textit{Proč?}
   	\item účelová \hfill \textit{Za jakým účelem?}
   	\item podmínková \hfill \textit{Za jakých podmínek?}
   	\item přípustková \hfill \textit{I přes co?}
    \end{itemize}
    \end{itemize}
    \end{itemize}
   	\item souvětí souřadné = spojení minimálně dvou vět hlavních
   	\item souvětí podřadné = spojení věty hlavní jednou enbo více větami vedlejšími
\end{itemize}

\section*{Větné členy}
\begin{itemize}
\item základní
\begin{itemize}
\item podmět: vyjádřený, nevyjádřený a všeobecný; holý, rozivý a několikanásobný
\item přísudek: slovesný jednoduchý (tvořen jedním slovesem v určité osobě), slovesný složený (tvořen způsobovým nebo fázovým slovesem a infinitivem); jmenný se sponou; jmenný beze spony; citoslovečný
\end{itemize}
\item rozvíjející
\begin{itemize}
\item přívlastek: shodný (shoduje se s podstatným jménem v pádě, čísle a rodě), neshodný (neshoduje se s podstatným jménem v pádě, čísle a rodě); volný (nezužuje význam řídícího podstatného jména, jen doplňuje, odděluje se čárkou), těsný (zužuje význam řídícího podsatného jména, neodděluje se čárkou), přístavek (základem je podstatné jméno nebo zájmeno, odděluje se čárkou z obou stran, pokud ale vyjadřuje funkci / titul / zaměstnání, čárkou se neodděluje); několikanásobný (několik přívlastků, jejich pořadí lze měnit, jsou odděleny čárkou nebo souřadící spojkou), postupně rozvíjející (pořadí členů není možné měnit, neoddělují se čárkou)
\item předmět
\item příslovečné určení: místa, času, způsobu, míry, příčiny, účelu, podmínky, přípustky
\item doplněk: závisí na dvou členech -- na slovesu a jméně
\end{itemize}
\end{itemize}

\section*{Slovotvorný a morfematický rozbor}
\begin{itemize}
	\item \textbf{morfematický rozbor} = rozklad na základní jednotky - \textit{morfémy}
	\item druhy morfémů
	\begin{enumerate}
		\item kořen - základní významová část slova \hfill\textit{les ve slově lesník}
		\item předpona - část před kořenem, mění nebo rozšiřuje význam slova \hfill\textit{vy- ve slově vypsat}
		\item přípona - část za kořenem, může měnit slovní druh a význam\hfill\textit{-ečí ve slově prasečí}
		\item koncovka - ohýbací část slova, nese mluvnické informace\hfill\textit{-a ve slově žena}
	\end{enumerate}
	\item příklad morfematického rozboru: představitelka = před- (předpona) stav (kořen) -itel (slovotvorná přípona) -k (přechylovací přípona - tvoří ženský rod) -a (koncovka prvního pádu jednotného čísla)
	\item kmen = slovo, ke kterému se přidává koncovka (např. městsk- v městský)
	\item postup při morfematickém rozboru
	\begin{enumerate}
		\item hledání kořene (hledáme co nejvíc slov ze stejné slovní čeledě a hledáme společné hlásky)
		\item grafické oddělení morfémů oddělení, případné označení nulové koncovky prvního pádu (např. u slova pán připíšeme na konec $\emptyset$)
	\end{enumerate}
	\item \textbf{slovotvorný rozbor} = analýza jak vzniklo slovo
	\item formant 
	\begin{itemize}
		\item[=] útvar, pomocí kterého se může tvořit nové slovo ze slova základového popř. z kořene
		\item formant může být předložka (pa-kůň), přípona (koň-sk(ý)), koncovka, zvratné sloveso (vrátit se), spojovací samohláska při skládání (zvěr\underline{o}kruh) 
	\end{itemize}
	\item slovotvorný základ 
	\begin{itemize}
		\item[=] část slova, ke kterému bylo něco přidáno a tím byl změněn význam
		\item může být kořen (např. les v lesník), základové slovo (prales v pralesní, sázen v sazenice), předložkové spojení (např. na břehu v nábřežní nebo před městem v předměstský)
	\end{itemize}
	\item postup slovotvorného rozboru
	\begin{itemize}
		\item Nalezení základového slova (např. uvážit v úvaha, ztěžknout v ztěžkl)
		\item Nalezení slovotvorného základu (např. úvah- v úvaha, ztežk- v ztěžkl)
		\item Zkoumání formantu (např. koncovka -a, prodloužení předponové samohlásky u- a změna hlásek h-ž v úvaha a koncovka -l v ztěžkl)
	\end{itemize}
\end{itemize}

\section*{Pravopis}
cvičení odhalte chyby, s/z

malá a velká písmena – prakticky – jaká jsou pravidla vysvětlit

jakoby x jako by: jakoby se dá nahradit jen jako, např. byl jakoby zmatený, když měním osobu z jako by bude jako bych

duál a duálové koncovky (očima, rukama, nohama, oběma dvěma…) – co to je


\section*{Slovní druhy}
	\subsection*{Zájména}
	\begin{itemize}
		\item druhy zájmen
		\begin{enumerate}
			\item osobní = zastupují osoby nebo věci\hfill\textit{já, ty, on, oni, mě, mu}
			\item přivlastňovací = vyjadřují, komu co patří\hfill\textit{moje, tvoje, náš, jejich}
			\item ukazovací = ukazují na osoby a věci\hfill\textit{ten, ta, tamten, onen, týž}
			\item tázací zájmena = slouží k tázání\hfill\textit{kdo, co, jaký, který, čí}
			\item vztažná zájmena = zavádějí vedlejší věty a odkazují na něco předchozího\hfill\textit{kdo, co, jaký, který, čí}
			\item neurčitá zájmena = odkazují na neurčitou osobu nebo věc\hfill\textit{někdo, něco, kdosi, cosi}
			\item záporná zájmena = vyjadřují zápor\hfill\textit{nikdo, nic, ničí}
			\item vyjadřují, že se děj vrátil ke svému původci\hfill\textit{se, si}
		\end{enumerate}
	\end{itemize}
	\subsection*{Číslovky}
	\begin{itemize}
		\item druhy číslovek
		\begin{enumerate}
			\item základní - vyjadřují počet\hfill\textit{jeden, dva, čtyři, padesát}
			\item řadové - vyjadřují pořadí\hfill\textit{první, třetí, sedmnáctý}
			\item druhové - vyjadřují počet skupin či druhů\hfill\textit{dvoje, čtvery}
			\item násobné - vyjadřují násobek\hfill\textit{jednou, pětkrát, stokrát}
			\item neurčité číslovky - opak určitých číslovek (1.-4.), neurčují přesné množství \hfill\textit{mnoho, málo, několikero, mnohokrát}
		\end{enumerate}
	\end{itemize}
	\subsection*{předložky}
	\begin{itemize}
		\item vlastní - můžou být jen předložky\hfill\textit{k, s, u}
		\item nevlastní - můžou být i jiné slovní druhy \hfill\textit{během, kolem}
		\item nepůvodní - vznikají z jiných slovních druhů, ale tvoří vlastní slovní celek\hfill\textit{s ohledem na, s přihlédnutím k, za účelem, pro potřeby}
	\end{itemize}

\section*{Mluvnické kategorie}
	\subsection*{Podstatná jména}
		\begin{itemize}
			\item pád (casus) - určuje syntaktickou funkci podstatného jména ve větě
			\item číslo (numerus) - singulár x plurál
			\item rod (genus) - střední, ženský, mužský životný, mužský neživotný
			\item vzor (deklinace) - určuje způsob skloňování
		\end{itemize}
	\subsection*{Přídavná jména}
		\begin{itemize}
			\item pád - shodné s podstatným jménem, ke kterému se váže
			\item číslo - shodné s podstatným jménem, ke kterému se váže
			\item rod - shodné s podstatným jménem, ke kterému se váže
			\item stupeň (komparace) - míra vlastnosti (pozitiv - chytrý, komparativ - chytřejší, superlativ - nejchytřejší)
			\item druh - tvrdý (různé tvary pro různé rody, skloňuje se podle vzoru mladý), měkký (stejné tvary pro různé rody, vzor jarní), přivlastňovací (vyjadřují vztah komu něco patří, vzor otcův a matčin)
		\end{itemize}
	\subsection*{Zájmena a číslovky}
		\begin{itemize}
			\item zájmena číslovky mají pád, rod a číslo, který funguje stejně jako u podstatných a přídavných jmen
			\item některá zájmena mají osobu (1. osoba - já, my můj, 2. osoba - ty, vy, tvůj, 3. osoba - on, ona, jeho, jejich)
			\item některé číslovky se stupňují (málo - méně - nejméně, mnoho - více - nejvíce)
		\end{itemize}
	\subsection*{Slovesa}
		\begin{itemize}
			\item osoba - určuje kdo vykonává děj (1. osoba - píšu, 2. osoba - píšete, 3. osoba - píše)
			\item číslo - určuje počet činitelů (singulár - dělá, plurál - děláme)
			\item čas - přítomný (čtu), minulý (četl jsem), budoucí (budu číst)
			\item způsob - oznamovací (on jde), rozkazovací (jdi), podmiňovací (šel bych)
			\item rod - rozlišuje, jestli podmět vykonává děj, nebo ho přijímá: činný (Petr napsal dopis), trpný (dopis byl napsán Petrem)
			\item vid - určuje zda je děj ukončený nebo neukončený: dokonavý (napsat, přečíst), nedokonavý (psát, číst)
			\item třída a vzor - není mluvnická kategorie, ale podle tříd se slovesa časují 
		\end{itemize}

\section*{Stupňování}
	\begin{itemize}
		\item přídavná jména označující vlastnosti (velký, malý, dobrý) se dají stupňovat ve třech stupních
		\begin{itemize}
			\item první stupeň (pozitiv) je základní tvar přídavného jména
			\item druhý stupeň (komparativ) se tvoří příponou -ší, -ejší nebo -í či změnou v kmeni, používá se při porovnávání dvou jevů
			\item třetí stupeň (superlativ) se tvoří předponou nej-, používá se při vyjádření dominance dané vlastnosti
		\end{itemize}
		\item příslovce vyjadřující způsob a míru se dají stupňovat do stejných stupňů jako přídavná jména (např. hezky - hezčeji - nejhezčeji)
		\item některé číslovky se dají také stupňovat (málo - méně - nejméně, mnoho - více - nejvíce)
	\end{itemize}

\section*{Slovotvorba}
a to patří pod to obohacování: skládání, odvozování, zkracování, značky

\textit{obohacování spisovné slovní zásoby} češtiny (přejímání z jiných jazyků, přejímání z nespisovné češtiny, slovotvorba, přenášení významů – nové významy na již existující slova ...)

\section*{Interpunkční znaménka a jejich funkce}
interpunkční znaménka a jejich funkce: čárky tečky pomlčky spojovníky, uvozovky, středníky

\section*{větná kondenzace}
větná kondenzace: “s ohledem na zastavení činnosti”, je to v administrativním, odborném stylu, často spojeno s nepůvodními předložkami, nahromadění především podstatných jmen, místo vedlejší věty, často ve druhém pádě, často podstatná jména, která jsou původně slovesa

\section*{útvary národního j}
útvary národního jazyka: spisovné x nespisovné, co do toho patří (vysoká formální

\section*{hláskosloví}
fonetika, fonologie, ortoepie, ortofonie x typy asimilací = spodob, koncová asimilace – had [hat], spodoba (asimilace) znělosti

\section*{univerbizace a multiverbizace}

\section*{lexikologie}
jednoduché x složené pojmenování, sousloví, historismy, archaismy, slova přejatá, expresivní výrazy, deminutiva = zdrobněliny, …

\section*{práce s citacemi}
práce s citacemi: co to je, proč se používají, jak se s nimi nejlépe pracuje (citace.com), psaní zdrojů, podle norem

\section*{druhy tázacích vět}
druhy tázacích vět: jednoduché, zjišťovací (ano/ne), doplňovací (otevřená odpověď)?

\section*{typografie}
použití kurzívy, tučného písma, podtrhávání, souvisí s publicistikou, velikost písma

\section*{Automatizace a aktu}
automatizace a aktualizace výrazu: co to je

\section*{enumerace}
enumerace (výčty) - proč to autoři někdy používají

\section*{druhy vět}
druhy vět podle postoje mluvčího: oznamovací, tázací, zvolací, rozkazovací

\section*{př spřežky}
příslovečné spřežky: spojení předložky a něčeho: dohromady x do hromady, skočil do hromady sena x skočili dohromady, na shledanou není spřežka!!!

\section*{anakolut, zeugma, kontaminace, atrakce}


\end{document}
