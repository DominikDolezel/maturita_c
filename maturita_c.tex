\documentclass{memoir}
\usepackage{fullpage}
\usepackage[czech]{babel}
\usepackage{amsmath}
\usepackage{amsfonts}
\usepackage{multirow}

\begin{document}
\section*{Stavba souvětí}
\begin{itemize}
\item poměry mezi souřadně spojenými větami
    \begin{itemize}
    \item slučovací (např. spojky \textit{a}, \textit{i}, \textit{ani})
    \item stupňovací (např. spojky \textit{dokonce, ba i})
    \item odporovací (např. spojky ale, přesto, avšak)
    \item vylučovací (např. spojky nebo, anebo)
    \item důvodový (např. spojky neboť, totiž, vždyť, přece)
    \item důsledkový (např. spojky proto, tedy, tudíž)
    \end{itemize}
    \item věta dvojčlenná = má podmět a přísudek \hfill \textit{Máme hodně úkolů.}
   	\item věta jednočlenná = má pouze přísudek, nemá podmět \hfill \textit{Prší. Je mi smutno.}
   	\item větný ekvivalent = nemá ani podmět, ani přísudek \hfill \textit{Nevstupovat!}
   	\begin{itemize}
    \item druhy vedlejších vět
    \begin{itemize}
    \item podmětná \hfill \textit{Kdo, co?}
   	\item přísudková \hfill \textit{Jaký byl/nebyl podmět?}
   	\item předmětná \hfill \textit{pádové otázky kromě 1. a 5. pádu}
   	\item přívlastková \hfill \textit{Jaký, který, čí?}
   	\item doplňková \hfill \textit{V jaké situaci?}
   	\item příslovečná
    \begin{itemize}
    \item místní \hfill \textit{Kde?}
   	\item časová \hfill \textit{Kdy?}
   	\item způsobová \hfill \textit{Jak?}
   	\item měrová \hfill \textit{Do jaké míry}
   	\item příčinná \hfill \textit{Proč?}
   	\item účelová \hfill \textit{Za jakým účelem?}
   	\item podmínková \hfill \textit{Za jakých podmínek?}
   	\item přípustková \hfill \textit{I přes co?}
    \end{itemize}
    \end{itemize}
    \end{itemize}
   	\item souvětí souřadné = spojení minimálně dvou vět hlavních
   	\item souvětí podřadné = spojení věty hlavní jednou nebo více větami vedlejšími
\end{itemize}

\section*{Větné členy}
\begin{itemize}
\item základní
\begin{itemize}
\item podmět: vyjádřený, nevyjádřený a všeobecný; holý, rozvitý a několikanásobný
\item přísudek: slovesný jednoduchý (tvořen jedním slovesem v určité osobě), slovesný složený (tvořen způsobovým nebo fázovým slovesem a infinitivem); jmenný se sponou; jmenný beze spony; citoslovečný
\end{itemize}
\item rozvíjející
\begin{itemize}
\item přívlastek: shodný (shoduje se s podstatným jménem v pádě, čísle a rodě), neshodný (neshoduje se s podstatným jménem v pádě, čísle a rodě); volný (nezužuje význam řídícího podstatného jména, jen doplňuje, odděluje se čárkou), těsný (zužuje význam řídícího podstatného jména, neodděluje se čárkou), přístavek (základem je podstatné jméno nebo zájmeno, odděluje se čárkou z obou stran, pokud ale vyjadřuje funkci / titul / zaměstnání, čárkou se neodděluje); několikanásobný (několik přívlastků, jejich pořadí lze měnit, jsou odděleny čárkou nebo souřadící spojkou), postupně rozvíjející (pořadí členů není možné měnit, neoddělují se čárkou)
\item předmět
\item příslovečné určení: místa, času, způsobu, míry, příčiny, účelu, podmínky, přípustky
\item doplněk: závisí na dvou členech -- na slovesu a jméně
\end{itemize}
\end{itemize}

\section*{Slovotvorný a morfematický rozbor}
\begin{itemize}
	\item \textbf{morfematický rozbor} = rozklad na základní jednotky - \textit{morfémy}
	\item druhy morfémů
	\begin{enumerate}
		\item kořen - základní významová část slova \hfill\textit{les ve slově lesník}
		\item předpona - část před kořenem, mění nebo rozšiřuje význam slova \hfill\textit{vy- ve slově vypsat}
		\item přípona - část za kořenem, může měnit slovní druh a význam\hfill\textit{-ečí ve slově prasečí}
		\item koncovka - ohýbací část slova, nese mluvnické informace\hfill\textit{-a ve slově žena}
	\end{enumerate}

	\item kmen = slovo, ke kterému se přidává koncovka (např. městsk- v městský)
	\item slovní čeleď = množina slov se stejným kořenem
	\item postup při morfematickém rozboru
	\begin{enumerate}
		\item hledání kořene (hledáme co nejvíc slov ze stejné slovní čeledě a hledáme společné hlásky)
		\item grafické oddělení morfémů oddělení, případné označení nulové koncovky prvního pádu (např. u slova pán připíšeme na konec $\emptyset$)
	\end{enumerate}
	\item příklad morfematického rozboru: představitelka = před- (předpona) stav (kořen) -itel (slovotvorná přípona) -k (přechylovací přípona - tvoří ženský rod) -a (koncovka prvního pádu jednotného čísla)\\
\end{itemize}
\begin{itemize}
	\item \textbf{slovotvorný rozbor} = analýza jak vzniklo slovo
	\item formant 
	\begin{itemize}
		\item[=] útvar, pomocí kterého se může tvořit nové slovo ze slova základového popř. z kořene
		\item formant může být předložka (pa-kůň), přípona (koň-sk(ý)), koncovka, zvratné sloveso (vrátit se), spojovací samohláska při skládání (zvěr\underline{o}kruh) 
	\end{itemize}
	\item slovotvorný základ 
	\begin{itemize}
		\item[=] část slova, ke kterému bylo něco přidáno a tím byl změněn význam
		\item může být kořen (např. les v lesník), základové slovo (prales v pralesní, sázen v sazenice), předložkové spojení (např. na břehu v nábřežní nebo před městem v předměstský)
	\end{itemize}
	\item postup slovotvorného rozboru
	\begin{itemize}
		\item Nalezení základového slova (např. uvážit v úvaha, ztěžknout v ztěžkl)
		\item Nalezení slovotvorného základu (např. úvah- v úvaha, ztežk- v ztěžkl)
		\item Zkoumání formantu (např. koncovka -a, prodloužení předponové samohlásky u- a změna hlásek h-ž v úvaha a koncovka -l v ztěžkl)
	\end{itemize}
\end{itemize}

\section*{Pravopis}
cvičení odhalte chyby, s/z

malá a velká písmena – prakticky – jaká jsou pravidla vysvětlit

jakoby x jako by: jakoby se dá nahradit jen jako, např. byl jakoby zmatený, když měním osobu z jako by bude jako bych

duál a duálové koncovky (očima, rukama, nohama, oběma dvěma…) – co to je


\section*{Slovní druhy}
	\subsection*{Zájména}
	\begin{itemize}
		\item druhy zájmen
		\begin{enumerate}
			\item osobní = zastupují osoby nebo věci\hfill\textit{já, ty, on, oni, mě, mu}
			\item přivlastňovací = vyjadřují, komu co patří\hfill\textit{moje, tvoje, náš, jejich}
			\item ukazovací = ukazují na osoby a věci\hfill\textit{ten, ta, tamten, onen, týž}
			\item tázací zájmena = slouží k tázání\hfill\textit{kdo, co, jaký, který, čí}
			\item vztažná zájmena = zavádějí vedlejší věty a odkazují na něco předchozího\hfill\textit{kdo, co, jaký, který, čí}
			\item neurčitá zájmena = odkazují na neurčitou osobu nebo věc\hfill\textit{někdo, něco, kdosi, cosi}
			\item záporná zájmena = vyjadřují zápor\hfill\textit{nikdo, nic, ničí}
			\item vyjadřují, že se děj vrátil ke svému původci\hfill\textit{se, si}
		\end{enumerate}
	\end{itemize}
	\subsection*{Číslovky}
	\begin{itemize}
		\item druhy číslovek
		\begin{enumerate}
			\item základní - vyjadřují počet\hfill\textit{jeden, dva, čtyři, padesát}
			\item řadové - vyjadřují pořadí\hfill\textit{první, třetí, sedmnáctý}
			\item druhové - vyjadřují počet skupin či druhů\hfill\textit{dvoje, čtvery}
			\item násobné - vyjadřují násobek\hfill\textit{jednou, pětkrát, stokrát}
			\item neurčité číslovky - opak určitých číslovek (1.-4.), neurčují přesné množství \hfill\textit{mnoho, málo, několikero, mnohokrát}
		\end{enumerate}
	\end{itemize}
	\subsection*{předložky}
	\begin{itemize}
		\item vlastní - můžou být jen předložky\hfill\textit{k, s, u}
		\item nevlastní - můžou být i jiné slovní druhy \hfill\textit{během, kolem}
		\item nepůvodní - vznikají z jiných slovních druhů, ale tvoří vlastní slovní celek\hfill\textit{s ohledem na, s přihlédnutím k, za účelem, pro potřeby}
	\end{itemize}

\section*{Mluvnické kategorie}
	\subsection*{Podstatná jména}
		\begin{itemize}
			\item pád (casus) - určuje syntaktickou funkci podstatného jména ve větě
			\item číslo (numerus) - singulár x plurál
			\item rod (genus) - střední, ženský, mužský životný, mužský neživotný
			\item vzor (deklinace) - určuje způsob skloňování
		\end{itemize}
	\subsection*{Přídavná jména}
		\begin{itemize}
			\item pád - shodné s podstatným jménem, ke kterému se váže
			\item číslo - shodné s podstatným jménem, ke kterému se váže
			\item rod - shodné s podstatným jménem, ke kterému se váže
			\item stupeň (komparace) - míra vlastnosti (pozitiv - chytrý, komparativ - chytřejší, superlativ - nejchytřejší)
			\item druh - tvrdý (různé tvary pro různé rody, skloňuje se podle vzoru mladý), měkký (stejné tvary pro různé rody, vzor jarní), přivlastňovací (vyjadřují vztah komu něco patří, vzor otcův a matčin)
		\end{itemize}
	\subsection*{Zájmena a číslovky}
		\begin{itemize}
			\item zájmena číslovky mají pád, rod a číslo, který funguje stejně jako u podstatných a přídavných jmen
			\item některá zájmena mají osobu (1. osoba - já, my můj, 2. osoba - ty, vy, tvůj, 3. osoba - on, ona, jeho, jejich)
			\item některé číslovky se stupňují (málo - méně - nejméně, mnoho - více - nejvíce)
		\end{itemize}
	\subsection*{Slovesa}
		\begin{itemize}
			\item osoba - určuje kdo vykonává děj (1. osoba - píšu, 2. osoba - píšete, 3. osoba - píše)
			\item číslo - určuje počet činitelů (singulár - dělá, plurál - děláme)
			\item čas - přítomný (čtu), minulý (četl jsem), budoucí (budu číst)
			\item způsob - oznamovací (on jde), rozkazovací (jdi), podmiňovací (šel bych)
			\item rod - rozlišuje, jestli podmět vykonává děj, nebo ho přijímá: činný (Petr napsal dopis), trpný (dopis byl napsán Petrem)
			\item vid - určuje zda je děj ukončený nebo neukončený: dokonavý (napsat, přečíst), nedokonavý (psát, číst)
			\item třída a vzor - není mluvnická kategorie, ale podle tříd se slovesa časují 
		\end{itemize}

\section*{Stupňování}
	\begin{itemize}
		\item přídavná jména označující vlastnosti (velký, malý, dobrý) se dají stupňovat ve třech stupních
		\begin{itemize}
			\item první stupeň (pozitiv) je základní tvar přídavného jména
			\item druhý stupeň (komparativ) se tvoří příponou -ší, -ejší nebo -í či změnou v kmeni, používá se při porovnávání dvou jevů
			\item třetí stupeň (superlativ) se tvoří předponou nej-, používá se při vyjádření dominance dané vlastnosti
		\end{itemize}
		\item příslovce vyjadřující způsob a míru se dají stupňovat do stejných stupňů jako přídavná jména (např. hezky - hezčeji - nejhezčeji)
		\item některé číslovky se dají také stupňovat (málo - méně - nejméně, mnoho - více - nejvíce)
	\end{itemize}

\section*{Slovotvorba a obohacování slovní zásoby}
	\begin{itemize}
		\item slova se třídí z hlediska slovotvorného na
		\begin{enumerate}
			\item neutvořená - postrádající slovotvorný prostředek\hfill\textit{ty, k, s, pst}
			\item utvořená kořenná - připnutí formantu (viz. slovotvorný rozbor) ke kořenu\hfill\textit{voda, těžký, koupit}
			\item utvořená nekořenná - vytvořená z jiného slova (odvozením, skládáním, zkracováním)\hfill\textit{prales, zeměkoule, Čedok}
		\end{enumerate}
		\item \textbf{odvozování} = tvorba slov na základě připojení formantu\hfill\textit{kovář$\rightarrow$kovárna}
		\item \textbf{skládání} = tvoření složeného slova z více základových slov\hfill\textit{noc + leh$\rightarrow$nocleh}
		\begin{itemize}
			\item složené slovo není slovo vzniklé z předložkového slovního spojení
			\item hybridní slovo je slovo vzniklé z cizího a českého slova\hfill\textit{elektrospotřebič}
		\end{itemize}
		\item \textbf{zkracování} 
		\begin{itemize}
			\item[=] zkrácení základového slova\hfill\textit{internát$\rightarrow$intr}
			\item[=] spojení několika částí základových slov\hfill\textit{$\underline{mo}$torový veloci$\underline{ped}$$\rightarrow$moped}
			\item[=] vznik z původní zkratky\hfill\textit{IKEM ze zkratky Institut klinické a expreimentální medicíny}
		\end{itemize}
		\item zkratková slova = slova vytvořená zkracováním, liší se od zkratek (USA, FSS), která nemají charakter slov\\
	\end{itemize}
 	\begin{itemize}
 		\item obohacování spisovné slovní zásoby může být pomocí
 		\begin{enumerate}
 			\item tvorbou nových slov (viz. výše)
 			\item přejímáním z cizích jazyků\hfill\textit{trénink, parfém, libreto}
 			\item přejímání z nespisovných vrstev češtiny\hfill\textit{holka, mobil, mail}
 			\item změna významů existujících slov\hfill\textit{pero, oko}
 		\end{enumerate}
 		\item přejímání z cizích jazyků
 		\begin{itemize}
 			\item kalk = doslovný překlad z cizího jazyka\hfill\textit{skyscraper$\rightarrow$mrakodrap}
 			\item přejímáme především z jazyků nám blízkým a pro nás kulturně relevantním (dříve němčina, latina, za první republiky francouzština, za socialismu ruština a nyní angličtina)
 		\end{itemize}
 		\item změna významu
 		\begin{enumerate}
 			\item rozšíření významu (generalizace)\hfill\textit{pero: ptačí brk$\rightarrow$psací náčinní}
 			\item zúžení významu (specializace)\hfill\textit{dojít: obecně přijít$\rightarrow$dojít pěšky}
 			\item přenesení významu (metafora): na základě podobnosti\hfill\textit{oko: oko sítě, oko polévky}
 			\item změna citového zabarvení
 			\begin{enumerate}
 				\item pozitivní posun (meliorace)\hfill\textit{pán: vlastník otroků$\rightarrow$zdvořilé oslovení}
 				\item negativní posun (pejorace)\hfill\textit{holota: chudý člověk$\rightarrow$hanlivě "lůza"}
 			\end{enumerate}
 		\end{enumerate}
 	\end{itemize}

\section*{Interpunkční znaménka a jejich funkce}
	\begin{enumerate}
		\item čárka (,)
		\begin{itemize}
			\item oddělují několikanásobné větné členy navzájem; jednotlivé věty v souvětí navzájem a některé polovětné konstrukce (těsný přívlastek); volně připojené větné členy (oslovení, citoslovce, vsuvky)
			\item věty závislé se oddělují od vět řídících
			\item věty spojené souřadně se oddělují buď čárkou nebo souřadící spojkou a, i, nebo, ani
			\item věty spojené jiným poměrem než souřadícím se oddělují čárkou
			\item vložená věta vedlejší a větná vsuvka musejí být odděleny čárkami z obou stran
		\end{itemize}
		\item středník (;)
		\begin{itemize}
			\item odděluje relativně samostatné významové celky uvnitř vět a souvětí; výčtové položky, uvnitř nichž je ještě třeba psát jiná interpunkční znaménka, či spojku; výčtové položky na různých řádcích
		\end{itemize}
		\item dvojtečka (:)
		\begin{itemize}
			\item uvozuje přímou řeč; uvozuje příklad, výčet čí vysvětlení předchozího textu
			\item po dvojtečce píšeme velké písmeno v případě přímé řeči nebo pokud část textu, kterou uvozuje má v sobě alespoň dvě samostatné věty oddělené tečkou
		\end{itemize}
		\item uvozovky („")
		\begin{itemize}
			\item uvozují přímou řeč, citát nebo ironické vyjádření
		\end{itemize}
		\item pomlčka (–)
		\begin{itemize}
			\item označuje nedokončenou výpověď, dramatickou pauzu, vysvětlení předchozího tvrzení, heslo, přímou řeč (místo uvozovek)
		\end{itemize}
		\item tři tečky (...)
		\begin{itemize}
			\item označují přerývavou, vzrušenou mluvu; nedokončenou výpověď; vynechání části citovaného textu; neúplnost výčtu
		\end{itemize}		
		\item závorky ( () )
		\begin{itemize}
			\item označují vsuvky, volně připojený větný člen, doplnění slova potřebného pro pochopení citátu či parafráze, autorství výroku
		\end{itemize}
		\item lomítko (/)
		\begin{itemize}
			\item označuje alternativu, pouze v publicistickém stylu 
		\end{itemize}
	\end{enumerate}

\section*{větná kondenzace}
větná kondenzace: “s ohledem na zastavení činnosti”, je to v administrativním, odborném stylu, často spojeno s nepůvodními předložkami, nahromadění především podstatných jmen, místo vedlejší věty, často ve druhém pádě, často podstatná jména, která jsou původně slovesa

\section*{Útvary národního jazyka}
	\begin{itemize}
		\item spisovná podoba národního jazyka
		\begin{enumerate}
			\item neutrální - uplatňující se v běžných a oficiálních komunikacích\hfill\textit{stůl, chodit, proč}
			\item hovorové - v ne příliš oficiálních situacích\hfill\textit{Pražák, mlíko, dneska, náklaďák}
			\item knižní - méně časté v mluveném projevu, hlavně ve vysoce oficiálním psaném projevu\hfill\textit{leč, neřkuli, bol}
			\item archaické - výrazy, které v současném jazyce silně zastarávají\hfill\textit{oř, jedvaže}
		\end{enumerate}
		\item nespisovné útvary jazyka
		\begin{enumerate}
			\item nářečí - jazykové útvary na určitých územích\hfill\textit{vokno, žaba, slepica}
			\item slang - jazykové útvary určitých povolání či zájmových skupin\hfill\textit{exnout, zemák, kule}
			\item argot (hantýrka) - jazykové útvary skupin na okraji společnosti\hfill\textit{čórnout, fízl}
		\end{enumerate}
	\end{itemize}

\section*{Hláskosloví}
	\begin{itemize}
		\item \textbf{fonetika} se zabývá hláskami z hlediska artikulačního a akustického, zkoumá stavbu a činnost mluvidel a místo a způsob artikulace
		\item \textbf{fonologie} se zabývá využitím zvukových jednotek. Zkoumá \textbf{fonémy}, které jsou schopny v daném jazyce vytvářet význam slov
		\item \textbf{ortofonie} vymezuje zásady správného vyslovování hlásek
		\item \textbf{ortoepie} stanovuje pravidla spisovné výslovnosti hlásek
		\item základní terminologie hláskosloví
		\begin{enumerate}
			\item mluvidla = orgány vytvářející artikulovanou řeč (dýchací, hlasové, artikulační)
			\item sluchové ústrojí = lidské ucho
			\item hláska = základní stavební prvek artikulované řeči, dělí se na
			\begin{enumerate}
				\item samohlásku (vokál) = hláska, při které se netvoří překážka vydechovacího proudu
				\item souhlásku (konsonant) = hláska, při které se tvoří překážka vydechovacímu proudu, její základ je šum
			\end{enumerate}
			\item dvojhláska = spojení dvou samohlásek v jedné slabice, jediná domácí samohláska je \textbf{ou}, v přejatých slovech i \textbf{au}, ojediněle i \textbf{eu} (euro)
			\item slabika = artikulační minimum, nejmenší jednotka řeči, kterou skutečně vyslovujeme. Slabiku mohou utvořit: samohláska, souhláska, dvojhláska nebo slabikotvorná souhláska (\textbf{l}, \textbf{r}, zřídka \textbf{m} a ojediněle \textbf{s}, \textbf{š})
			\item foném 
			\begin{itemize}
				\item[=] hláska se schopností měnit význam slova (např. p a š ve slově pila - šila)
				\item někdy mají fonémy více hlásek (např. n se vyslovuje jinak ve slove Hana a Hanka), zatímco v němčině se jedná o dva samostatné fonémy (sinnen - singen)
				\item některé fonémy mají znělou a neznélou variatnu (ř, c, č)
			\end{itemize}
		\end{enumerate}
	\item fonetická transkripce = transkripce slov podle mezinárodně dohodnutého systému hlásek (IPA). Např. přepis pěšec na [pješec] nebo vztah na [fstach]
	\item samohlásky dělíme podle místa artikulace (vepředu / vzadu, vysoko / nízko) a délky (krátké / dlouhé)
	\item souhlásky jsou buď znělé nebo neznělé, některé mají svoji znělou resp. neznělou dvojici dle tabulky
	\begin{table}[h]
		\centering
		\begin{tabular}{|l|l|l|l|l|l|l|l|l|l|l|l|}
			\hline
			\multirow{2}{*}{Párové} & neznělé & p & t & ť & k & f & s & š & ch & c  & č  \\ 
			
			& znělé   & b & d & ď & g & v & z & ž & h  & dz & dž\\ 
			\hline
			Nepárové                & znělé   & m & n & ň & l & r & j &   &    &    &    \\ 
			\hline
		\end{tabular}
		\caption{tabulka souhlásek podle znělosti a neznělosti se zvýrazněním párových souhlásek}
	\end{table}
	\item \textbf{spodoba hlásek} (asimilace) je přizpůsobení výslovnosti hlásky, aby se aritkulovali podobně. Týká se to pouze párových souhlásek, ty se poté mění na svoji druhou variantu.
	\begin{itemize}
		\item Znělá + Neznělá $rightarrow$ Znělá + Znělá\hfill\textit{ať dá [aď dá]}
		\item Neznělá + Znělá $rightarrow$ Neznělá + Neznělá\hfill\textit{pod kopcem [pot kopcem]}
		\item Neznělá + Nepárová $rightarrow$ Znělá + Nepárová\hfill\textit{ať je [aď je]}
	\end{itemize}
	\item neutralizace protikladu znělosti hlásek před pauzou (třeba koncem věty), znělá souhláska se mění na neznělou např. To je had. [hat]
	\end{itemize}

\section*{Univerbizace a multiverbizace}
	\begin{itemize}
		\item \textbf{univerbizace} je proces, kdy se ze sousloví stane jedno slovo\hfill\textit{jízdní kolo$\rightarrow$kolo, tělesná výchova$rightarrow$tělocvik}
		\item univerbáty jsou úspornější a lépe se přizpůsobují gramatice, užívají se v běžné i profesionální mluvě, některé mají hovorový charakter (vysoká škola$\rightarrow$vejška)
		\item \textbf{multiverbizace} je typ sousloví utvořen z jediného slova\hfill\textit{měřit$\rightarrow$provést měření}
		\item multiverbáty se užívají v publicistickém a administrativním stylu k dosažení většího přesnosti a jednodušejí se rozvíjí slovní spojení (třikrát popravil$\rightarrow$vykonal tři popravy)
	\end{itemize}
	
\section*{Lexikologie}
	\begin{itemize}
		\item \textbf{pojmenování} = přiřazení k určitému jevu, předmětu, vlastnosti (i abstraktnímu) tzv. jazykový výraz - v mluveném projevu hláskový, v psaném grafický
		\item způsoby pojmenování
		\begin{enumerate}
			\item slovo = základní jednotka slovní zásoby, je znak, který pojmenovává určitý předmět, jev, děj, vlastnost nebo vztah mezi jednotkami slovní zásoby (např. předkložky)
			\item složené pojmenování = volné spojení slov, která spolu mají pojmenovávací funkci, avšak si zachovávají plně věcný význam (např. tropická květina, čestné slovo)
			\item sousloví je spojení více slov k označení jiného pojmu, jednotlivá slova nemají týž význam samotná jako v sousloví (např. kyselina solná, ústřední topení)
			\item frazém (idiom) = ustálení spojení dvou a více slov k označení jednoho pojmu (např. na vlastní kůži, z bláta do louže)
		\end{enumerate}
	\end{itemize}
jednoduché x složené pojmenování, sousloví, historismy, archaismy, slova přejatá, expresivní výrazy, deminutiva = zdrobněliny, …

\section*{práce s citacemi}
	\begin{itemize}
		\item citace je přesné převzetí části cizího textu (nebo odkazu na něj)
		\item buď je přímá (v uvozovkách) nebo nepřímá (parafráze)
		\item citace je nutné uvést v bibliografii např. podle normy ČSN ISO 690 (převážně na konci), pomůcka citace.com
	\end{itemize}

\section*{druhy tázacích vět}
	\begin{itemize}
		\item zjišťovací otázky = cíl získat odpověď ano nebo ne\hfill\textit{Půjdeš do kina?}
		\item doplňovací otázky = cíl získat konkrétní informaci\hfill\textit{Kde bydlíš?}
		\item vylučovací otázky = nabízejí výběr mezi dvěma nebo více možnostmi\hfill\textit{Dáš si kávu nebo čaj?}
		\item řečnické otázky = otázky, na které tazatel nečeká odpověď\hfill\textit{Kdo by to čekal?}
	\end{itemize}

\section*{typografie}
použití kurzívy, tučného písma, podtrhávání, souvisí s publicistikou, velikost písma

\section*{Automatizace a aktualizace}
	\begin{itemize}
		\item \textbf{automatizace} = slovní výraz ztrácí svou původní výraznost, stává se "jazykovou rutinou"\hfill\textit{Přeji hezký den}
		\item \textbf{aktualizace} = výraz je použit nově originálně, aby upoutal pozornost\hfill\textit{šel domů$\rightarrow$odplazil se do svého brlohu} 
		\item využívá se v publicistice a beletrii
	\end{itemize}

\section*{enumerace}
enumerace (výčty) - proč to autoři někdy používají

\section*{druhy vět}
druhy vět podle postoje mluvčího: oznamovací, tázací, zvolací, rozkazovací

\section*{př spřežky}
příslovečné spřežky: spojení předložky a něčeho: dohromady x do hromady, skočil do hromady sena x skočili dohromady, na shledanou není spřežka!!!

\section*{anakolut, zeugma, kontaminace, atrakce}


\end{document}
